\documentclass[10pt]{article}
\usepackage[margin=2cm]{geometry}
\usepackage[ngerman]{babel}

\usepackage[inkscapeformat=png]{svg}
\usepackage{fancyhdr}

\usepackage{lastpage}

\usepackage{inter}
\renewcommand\familydefault{\sfdefault}

\usepackage[onehalfspacing]{setspace}
\usepackage[iso,german]{isodate}

\usepackage{xcolor}
\usepackage{hyperref}

\setcounter{tocdepth}{1}
\setlength\parindent{0pt}

\usepackage{titlesec}
\titleformat{\subsection}[runin]{\normalfont\bfseries}{\thesubsection}{0.5em}{}[]
\titleformat{\subsubsection}[runin]{\normalfont\bfseries}{\thesubsubsection}{0.5em}{}[]

\def\Year{\expandafter\YEAR\the\year}
\def\YEAR#1#2#3#4{#1#2#3#4}

\fancyhead{}
\pagestyle{fancy}
\renewcommand{\headrulewidth}{0pt}
\fancyfoot{}
\fancyfoot[L]{\textcolor{gray}{\thepage\space/\space\pageref{LastPage}}}
\fancyfoot[R]{\textcolor{gray}{\href{https://helpwave.de}{helpwave.de}}}

\begin{document}

\begin{center}
	\includesvg[width=0.2\linewidth]{logo.svg}\\[4ex]
	{\Huge \sffamily
	Nutzungsbedingungen \\
	\huge App zum Doc \\[1ex] \Large Erbringer-Anwendung}\\[2ex]

	\vspace{1em}
	{\large
		helpwave GmbH \\
		Jülicher Straße 209d \\
		52070 Aachen \\
		HRB 27480}\\
	\vspace{1em}
	Stand: \today
\end{center}

\vspace{2ex}

\tableofcontents

\newpage

\section{Verwendete Begrifflichkeiten}
„Nutzer“ im Sinne dieser Nutzungsbedingungen sind Personen, die die App zum Doc über von einen dafür vorgesehen Weg heruntergeladen und auf einem Endgerät installiert haben und ihre Funktionionen zur Kommunikation mit Erbringern, d.h. Dienstleistungsanbietern im Gesundheitsweisen, die ihrerseits die App zum Doc nutzen, verwenden.

\section{Geltung der Nutzungsbedingungen}
Diese Nutzungsbedingungen gelten für alle von helpwave im Kontext der App zum Doc den Nutzern zur Verfügung gestellten Dienstleistungen und Funktionen. Die Nutzer müssen die App zum Doc ordnungsgemäß auf einem der offiziel durch helpwave zur Verfügung gestellten Wege heruntegeladen und auf einem der dafür vorgesehenen Endgeräte ordnungsgemäß installiert haben. Der Zugriff und die Nutzung der App zum Doc setzt voraus, dass die Nutzer die vorliegenden Nutzungsbedingen gelesen und ihnen zugestimmt haben. Die Nutzungsbedingungen werden im Rahmen der erstmaligen Inbetriebnahme der App zum Doc dem Nutzer auf dem mobilen Endgerät angezeigt. Die Inbetriebnahme kann ohne explizite Zustimmung des Nutzers zu den Nutzungsbedingungen nicht erfolgreich abgeschlossen werden. Die Nutzungsbedingungen werden dem Nutzer dauerhaft in dem dafür vorgesehen Bereich der App zum Doc oder einem darüber erreichbaren, dauerhaft verfügbaren Ort zur Verfügung gestellt und können dort vom Nutzer auch heruntergeladen werden.

\section{Zustandekommen des Nutzungsvertrags}
Der Nutzer schließt den Vertrag über die Nutzung der Dienste von helpwave mit der helpwave GmbH. Der Vertragsschluss kommt durch die erstmalige Inbetriebnahme der App zum Doc nach Bestätigung der Nutzungsbedingungen zustande. Die Funktionen der App zum Doc werden erst nach Ablauf der Widerrufsfrist freigeschaltet. Der Nutzer hat jedoch im Rahmen der erstmaligen Inbetriebnahme die Möglichkeit, von helpwave ausdrücklich mit dem Beginn der Dienstleistungen vor Ablauf der Widerrufsfrist zu verlangen. Er wird in diesem Zug darauf hingewiesen, dass damit sein Widerrufsrecht erlischt.

\section{Widerruf und Kündigung des Nutzungsvertrages durch die Nutzer}
Der Nutzer hat das Recht, den Nutzungsvertrag innerhalb von 14 Tagen ohne Angabe von Gründen zu widerrufen. Die Widerrufsfrist beginnt mit Vertragsschluss. Für die Ausübung des Widerrufs muss der Nutzer der helpwave GmbH den Widerruf per Post oder Email oder unter Nutzung des Widerrufsformular erklären. Die Absendung des Widerrufs innerhalb der Widerrufsfrist reicht zur Wahrung der Frist. Folgen des Widerrufs sind die Beendigungen der Dienstleistungen der helpwave GmbH im Kontext der App zum Doc und die endgültige Löschung der mit dem Nutzer verbunden und durch ihn eingefügten Daten, vorbehaltlich derjeningen Daten, die aufgrund einer gesetzlichen Pflicht aufbewahrt werden müssen.
Der Nutzer hat zudem die Möglichkeit, jederzeit und ohne Einhaltung einer Kündigungsfrist, den Nutzungsvertrag ohne Angabe von Gründen zu kündigen. Die Kündigung kann durch explizite Willensäußerung oder durch die Löschung des Nutzungskontos erfolgen. Folge der Kündigung ist ebenfalls die endgültige Löschung der mit dem Nutzer verbundenen und durch ihn eingefügten Daten, vorbehaltlich der Daten, die aufgrund einer gesetzlichen Pflicht aufbewahrt werden müssen.

\section{Art und Umfang der App zum Doc}
helpwave stellt Nutzern die App zum Doc zur Verfügung. Nutzer können ihr Profil anlegen. Zu den Informationen, die
in diesem Profil vorhanden sind, zählen der Name, Geburtsdatum, Telefonnummer, Art und Anbieter der Krankenversicherung, Wohnort und optional die Versicherungsnummer. Es können mehrere Profile pro Nutzer angelegt werden. In der App zum Doc können Nutzer mit Erbringern verschlüsselt kommunizieren (Chat), Bilder an Erbringer senden sowie Informationen über ebenfalls in der App zum Doc angemeldete Erbringer sowie allgemeine Informationen (zB Ärztlicher Notdienst, etc.) einholen. Sie können Medikationspläne übermitteln (lassen) sowie Termine, Rezepte oder Überweisungen anfragen. Bearbeitete Anfragen verbleiben bis zur endgültigen Löschung durch den Erbringer verschlüsselt in einem Archiv.
Damit die Kommunikation zwischen Nutzer und Erbringer verschlüsselt stattfindet, wird neben der SSL-verschlüsselten
Übertragung der Anfragen auch eine Verschlüsselung dieser Anfrage nach dem Public-Private-Key-Verfahren durchgeführt.
Hierzu erzeugt der Erbringer auf seinem Endgerät bei der Ersteinrichtung seines Profils einen privaten und einen öffentlichen Schlüssel. Der öffentliche Schlüssel wird an die Systeme von helpwave übermittelt und dort gespeichert. Anfragen von
Nutzern werden mit diesem Schlüssel automatisch verschlüsselt und erst danach an die Systeme von helpwave übertragen.
Der private Schlüssel wird nur auf dem Endgerät des Erbringers erzeugt und ihm dort zum Speichern angeboten. Er wird zu
keinem Zeitpunkt durch die Erbringer-Anwendung an die Systeme von helpwave übertragen.



\section{Nutzungsgebühren}
Die Nutzung der App zum Doc ist für den Nutzer kostenlos.

\section{Abfrage und Änderung von personenbezogenen Daten durch den Erbringer}
Nutzer, die ihr Profil in der App löschen möchten, können dies erreichen, indem sie den mit helpwave abgeschlossenen
Nutzungsvertrag kündigen.

\section{Datenschutz}
Zur Nutzung der App zum Doc ist eine Eingabe personenbezogener Daten erforderlich, die den Nutzer gegen-
über den Erbringern eindeutig identifiziert. Hierbei handelt es sich um den Namen des Erbringers, sein Geburtsdatum, die Telefonnummer, sein Wohnort, die Krankenversicherung und optional die Versicherungsnummer.
Mit einer Anfrage eines Nutzers bei einem Erbringer werden die Anfragedaten in verschlüsselter Form an den Erbringer
übermittelt (Ende-zu-Ende-Verschlüsselung) und dort ebenso in verschlüsselter Form gespeichert. Antworten des Erbringers werden in gleicher Weise verschlüsselt. Die Verschlüsselung erfolgt während des gesamten Prozesses durch ein Public-
Private-Key-Verfahren, bei dem der private Schlüssel sich ausschließlich im Besitz des Absenders/Empfängers befindet. Die
Verschlüsselung erfolgt mit mindestens 2048-Bit (asymmetrisch) und mindestens 56-Bit (symmetrisch). Alle Übertragungen
erfolgen zudem über eine gesicherte SSL-Verbindung. \\

Weitere Hinweise zum Datenschutz finden Sie hier: \href{https://cdn.helpwave.de/privacy.html}{cdn.helpwave.de/privacy.html}

\section{Zulässige Nutzung}
Die App zum Doc sowie alle dazu gehörigen Websites (app-zum-doc.de) sowie zugehörige Alias-Domänen,
die auf app-zum-doc.de weiterleiten, dürfen ausschließlich für die folgenden Zwecke genutzt werden:
(1.) Anzeigen der Webseite, (2.) Prüfen von Informationen zu dem Leistungsangebot, (3.) Bearbeitung und Beantwortung von Anfragen oder Antworten der Erbringer sowie Kommunikation mit diesen (4.) Nutzung weiterer Funktionen, die auf der Webseite verfügbar gemacht werden. Jede andere Nutzung der Webseite / Erbringer-Anwendung ohne vorherige schriftliche Zustimmung der helpwave ist untersagt. Die Vervielfältigung, Bearbeitung, Verbreitung, öffentliche Wiedergabe oder jede sonstige Form der Nutzung von Inhalten der Webseite / Erbringer-Anwendung zu gewerblichen Zwecken ist unzulässig. Ebenso untersagt ist die Verwendung automatisierter Systeme oder automatisierter Software zur Extraktion von Inhalten von der Webseite / App zum Doc. Unbeschadet der Geltendmachung sonstiger Rechte behält sich helpwave vor, den Zugang zur Webseite / App zum Doc jederzeit zu sperren, wenn gegen diese Nutzungsbedingungen verstoßen wird.

\section{Pflichten des Nutzers}
\begin{itemize}
	\item Der Nutzer verpflichtet sich, seine Zugangsdaten zur Oberfläche für die Bearbeitung von Nutzeranfragen streng vertraulich zu behandeln und nicht weiterzugeben. Dies gilt sowohl für den Benutzernamen und das Kennwort, als auch für die individuelle Schlüsseldatei zur Ver- und Entschlüsselung der Anfragen.
	\item Der Nutzer verpflichtet sich, über die App zum Doc keine Inhalte weiterzuleiten, die rechtswidrig, insbesondere
	      rassistisch, pornografisch, beleidigend oder verleumderisch sind oder die Rechte Dritter, insbesondere Urheberrechte bzw.
	      urheberrechtliche Nutzungsrechte verletzen.
	\item Der Nutzer ist verpflichtet, niemanden mit Kommunikationsversuchen zu belästigen.
	\item Der Nutzer hat alle Handlungen zu unterlassen, die geeignet sind, die Funktionalität der App zum Doc zu beeinträchtigen, insbesondere übermäßig zu belasten.
	\item Der Nutzer trägt dafür Sorge, sämtliche Systeme und Endgeräte, mit denen auf Inhalte und Kommunikationsfunktionen der Webseite / App / App zum Doc zugegriffen wird, durch geeignete Schutzmaßnahmen vor unberechtigten
	      Zugriffen zu schützen.
\end{itemize}

\section{Änderung des Leistungsumfangs}
helpwave behält sich vor, die angebotene Anwendung und zugehörige Dienste jederzeit inhaltlich, grafisch und/oder funktionell durch Updates zu verändern oder zu erweitern.

\section{Zeitliche Verfügbarkeit}
helpwave wird sich bemühen, die angebotene Anwendung und zugehörige Dienste möglichst unterbrechungsfrei zum Abruf
anzubieten. Der Nutzer erkennt an, dass helpwave auch bei aller Sorgfalt eine zeitlich ununterbrochene Verfügbarkeit
seiner Dienste technisch nicht garantieren kann. Insbesondere behält sich helpwave vor, seine Dienste aus Wartungs-, Sicherheits- oder Kapazitätsgründen nach dem Ermessen von helpwave vorübergehend einzuschränken oder auszusetzen. helpwave wird den MNutzer soweit möglich über größere geplante Wartungsarbeiten (länger als 30 Minuten) über allgemein zugängliche Informationskanäle im Voraus informieren.

\section{Haftung und Gewährleistungsausschluss von helpwave}
helpwave haftet nur bei Vorsatz und grober Fahrlässigkeit. Bei leichter Fahrlässigkeit haftet helpwave nur für Verletzungen einer wesentlichen vertraglichen Verpflichtung, die erforderlich ist, damit der Vertrag ordnungsgemäß umgesetzt werden kann und auf die sich der Nutzer normalerweise verlassen kann. Die Haftung ist auf die vorhersehbaren und vertragstypischen Schäden beschränkt. Insbesondere wird die Haftung für Datenverlust auf den typischen Wiederherstellungsaufwand beschränkt, der bei regelmäßiger und gefahrentsprechender Anfertigung von Sicherungskopien eingetreten wäre. Die unter vorstehendem Absatz enthaltenen Haftungsbeschränkungen umfassen auch etwaige Ansprüche des Nutzers auf den Ersatz solcher Aufwendungen, die er im Vertrauen auf den Erhalt einer vertragsgemäßen Leistung gemacht hat und billigerweise machen durfte.

Hingegen haftet helpwave gegenüber Nutzern für Schäden aus der Verletzung des Lebens, des Körpers
oder der Gesundheit. Darüber hinaus haftet helpwave für Schäden, die von der Haftung nach zwingend
geltenden gesetzlichen Vorschriften umfasst werden sowie im Fall der etwaigen Übernahme einer Garantie durch helpwave.

\section{Geistiges Eigentum}
Alle Urheber-, insbesondere Datenbank-, Marken-, Geschmacksmuster- und andere Rechte zum Schutze geistigen Eigentums
(ebenso die Anordnung und die Darstellung der Webseite) liegen bei helpwave oder ihren Lizenzgebern. Ohne die vorherige
schriftliche Zustimmung dürfen die Inhalte der Webseite oder der zugrunde liegende Software-Code weder ganz oder teilweise vervielfältigt, bearbeitet, verbreitet, öffentlich wiedergegeben noch in sonstiger Form genutzt werden.

\section{Sonstiges}
helpwave behält sich vor, diese Nutzungsbedingungen jederzeit ohne Nennung von Gründen zu verändern oder zu erweitern,
es sei denn, das ist für den Erbringer nicht zumutbar. helpwave wird den Nutzer über jede Änderung der Nutzungsbedingungen rechtzeitig benachrichtigen. Der Nutzer kann den geändertern Nutzungsbedinungen innerhalb einer Frist von sechs (6) Wochen nach der Benachrichtigung zustimmen. helpwave wird den Nutzer in der Benachrichtigung auf sein Widerspruchsrecht und die Bedeutung der Widerspruchsfrist hinweisen. Stimmt der Nutzer den geänderten Nutzungsbedingungen nicht innerhalb der genannten Frist zu, gilt der Nutzungsvertrag als beendet. Soweit nichts anderes vereinbart ist, kann der Nutzer alle Erklärungen an helpwave per E-Mail oder per Brief an helpwave
übermitteln. helpwave kann Erklärungen gegenüber dem Nutzer per E-Mail übermitteln, die der
Erbringer als aktuelle Kontaktdaten in seinem Konto angegeben hat.
Sollten einzelne Regelungen dieser Nutzungsbedingungen unwirksam sein oder werden, wird dadurch die Wirksamkeit der
übrigen Regelungen nicht berührt. \\
Erfüllungsort ist der Sitz von helpwave.
Es gilt deutsches Recht.

\end{document}
