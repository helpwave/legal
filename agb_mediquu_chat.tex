\documentclass[10pt]{article}
\usepackage[margin=2cm]{geometry}
\usepackage[ngerman]{babel}

\usepackage[inkscapeformat=png]{svg}
\usepackage{fancyhdr}

\usepackage{lastpage}

\usepackage{inter}
\renewcommand\familydefault{\sfdefault}

\usepackage[onehalfspacing]{setspace}
\usepackage[iso,german]{isodate}

\usepackage{xcolor}
\usepackage{hyperref}

\setcounter{tocdepth}{1}
\setlength\parindent{0pt}

\usepackage{titlesec}
\titleformat{\subsection}[runin]{\normalfont\bfseries}{\thesubsection}{0.5em}{}[]
\titleformat{\subsubsection}[runin]{\normalfont\bfseries}{\thesubsubsection}{0.5em}{}[]

\def\Year{\expandafter\YEAR\the\year}
\def\YEAR#1#2#3#4{#1#2#3#4}

\fancyhead{}
\pagestyle{fancy}
\renewcommand{\headrulewidth}{0pt}
\fancyfoot{}
\fancyfoot[L]{\textcolor{gray}{\thepage\space/\space\pageref{LastPage}}}
\fancyfoot[R]{\textcolor{gray}{\href{https://helpwave.de}{helpwave.de}}}

\begin{document}

\begin{center}
	\includesvg[width=0.2\linewidth]{logo.svg}\\[4ex]
	{\Huge \sffamily
	Nutzungsbedingungen \\
	\huge mediQuu Team: Chat}\\[2ex]

	\vspace{1em}
	{\large
		helpwave GmbH \\
		Jülicher Straße 209d \\
		52070 Aachen \\
		HRB 27480}\\
	\vspace{1em}
	Stand: \today
\end{center}

\vspace{2ex}

\tableofcontents

\newpage

\section{Verwendete Begrifflichkeiten}
„Nutzer“ im Sinne dieser Nutzungsbedingungen sind Leistungserbringer im Gesundheitswesen (in der Regel Ärzte, Apotheken oder Mitglieder eines Praxisnetzes), die über mediQuu Chat mit anderen Nutzern kommunizieren. Der Nutzer verwendet
für diese Zwecke die Anwendung helpwave Team für mobile und stationäre Einsatzzwecke (gemeinschaftlich im Folgenden
als „Anwendung“ bezeichnet).

\section{Geltung der Nutzungsbedingungen}
Diese Nutzungsbedingungen, die gelegentlichen Veränderungen unterliegen, gelten für alle helpwave-Dienstleistungen
rund um die Anwendung, die unmittelbar oder mittelbar (d.h. über Dritte) über das Internet, jegliche Art von mobilen Endgeräten oder per E-Mail zur Verfügung gestellt werden.
Mit dem Zugriff auf die und der Nutzung der Anwendung, ganz gleich durch welche Plattform (im Folgenden die „Plattform“
genannt), bestätigt der Nutzer, dass er diese Nutzungsbedingungen gelesen und verstanden hat und diesen zustimmt.

\section{Zustandekommen des Nutzungsvertrags}
Der Nutzer schließt den Vertrag über die Nutzung der Anwendung mit der helpwave GmbH.
Durch den webbasierten Gebrauch der Software erklärt sich der Nutzer mit diesen Nutzungsbedingungen einverstanden und
erkennt diese ohne Einschränkung verbindlich an.

\section{Art und Umfang der Anwendung}
helpwave stellt Nutzern über eine mobile und/oder eine stationäre Anwendung eine Lösung zur Verfügung, die es ermöglicht, in einem Chat-Bereich mit anderen autorisierten Teilnehmern zu kommunizieren. Die App steht für iOS- und Androidgeräte, sowie PC- und Mac-Computer in den entsprechenden Stores und über einen Downloadlink zur Verfügung.
Nach der Registrierung für das Produkt und der Installation auf einem Smartphone oder der Nutzung auf einem PC/Mac
kann der Nutzer sich mit den ihm mitgeteilten Zugangsdaten anmelden. Erst danach haben autorisierte Teilnehmer Zugriff
auf ein Adressbuch, das das Hinzufügen von Gesprächspartnern ermöglicht für 1:1- oder 1:n-Chats. Der Austausch kann per
Text- und Bildnachricht stattfinden. Alle übermittelten Daten werden verschlüsselt übertragen. Die technische Infrastruktur
befindet sich in einem zertifizierten, deutschen Rechenzentrum.

\section{Nutzungsgebühren}
Die Nutzung der Anwendung ist für den Nutzer kostenpflichtig und erfordert einen gesonderten Vertrag zwischen helpwave
und dem jeweiligen Nutzer (Nutzungsvertrag).
helpwave stellt dem Erbringer die Nutzung der Erbringer-Anwendung gegen eine monatliche Lizenzgebühr zur Verfügung. helpwave behält sich Gebührenänderungen vor. Geänderte Gebühren, die für das jeweilige Vertragsverhältnis entscheidend sind, werden dem Nutzer mindestens sechs Wochen vor Inkrafttreten per E-Mail oder postalisch mitgeteilt. Preiserhöhungen bewirken ein außerordentliches Kündigungsrecht mit einer Frist von vier Wochen zum Zeitpunkt des Inkrafttretens der Erhöhung. Die Rechnungen von helpwave sind innerhalb von 14 Tagen und ohne Abzug nach Rechnungseingang zu überweisen. Eine Aufrechnung des Kunden gegenüber Forderungen von helpwave mit anderen als unbestritten oder rechtskräftig festgestellten Forderungen ist ausgeschlossen, es sei denn, die Gegenforderung und die aufgerechnete Hauptforderung sind synallagmatisch miteinander verknüpft. Befindet sich der Kunde mit dem Ausgleich einer Forderung im Verzug, ist helpwave berechtigt, Verzugszinsen in Höhe von drei Prozentpunkten über dem Basiszinssatz gemäß § 288 BGB zu verlangen. Die Geltendmachung eines weiteren Verzugsschadens bleibt davon unberührt.

\section{Abfrage und Änderung von personenbezogenen Daten durch den Nutzer}
Nutzer, die Ihr Konto für die Chat-Anwendung löschen möchten, können dies erreichen, indem sie den mit helpwave abgeschlossenen Nutzungsvertrag kündigen.

\section{Datenschutz}
Zur Nutzung der Anwendung ist eine Eingabe personenbezogener Daten erforderlich, die den Nutzer gegenüber anderen
Nutzern eindeutig identifizieren. Hierbei handelt es sich um den Vor- und Nachnamen des Nutzers. Bei einem Versand werden die Daten in verschlüsselter Form an den Empfänger übermittelt und erst dort entschlüsselt (Ende-zu-Ende-Verschlüsselung). Die Verschlüsselung erfolgt während des gesamten Prozesses durch ein Public-Private-Key-Verfahren, bei dem der private Schlüssel sich ausschließlich im Besitz des Absenders/Empfängers befindet. Die Verschlüsselung erfolgt mit mindestens
2048-Bit (asymmetrisch) und mindestens 56-Bit (symmetrisch). Alle Übertragungen erfolgen zudem über eine gesicherte
SSL-Verbindung. \\

Weitere Hinweise zum Datenschutz finden Sie hier: \href{https://cdn.helpwave.de/privacy.html}{cdn.helpwave.de/privacy.html}

\section{Zulässige Nutzung}
Die Anwendung, die dazugehörige Webseite (chat.mediquu.de) sowie zugehörige Alias-Domänen, die auf chat.mediquu.de
weiterleiten, dürfen ausschließlich für die folgenden Zwecke genutzt werden: (1.) Anzeigen der Anwendung, (2.) Prüfen von
Informationen zu dem Leistungsangebot, (3.) Kommunikation mit anderen Nutzern (4.) Nutzung weiterer Funktionen, die
in der Anwendung verfügbar gemacht werden. Jede andere Nutzung der Webseite / Anwendung ohne vorherige schriftliche Zustimmung von helpwave ist untersagt. Die Vervielfältigung, Bearbeitung, Verbreitung, öffentliche Wiedergabe oder
jede sonstige Form der Nutzung von Inhalten der Webseite / Anwendung zu gewerblichen Zwecken ist unzulässig. Ebenso
untersagt ist die Verwendung automatisierter Systeme oder automatisierter Software zur Extraktion von Inhalten von der
Webseite / Anwendung. Auch ist jeder Zugriff auf Inhalte der Webseite / Anwendung unzulässig, der nicht über die Benutzeroberfläche der Webseite / Anwendung erfolgt.
Unbeschadet der Geltendmachung sonstiger Rechte behält sich helpwave vor, den Zugang zur Webseite / Anwendung jederzeit zu sperren, wenn gegen diese Nutzungsbedingungen verstoßen wird.

\section{Pflichten des Nutzers}
\begin{itemize}
	\item Der Nutzer verpflichtet sich, seine Zugangsdaten zur Anwendung streng vertraulich zu behandeln und nicht weiterzugeben.
	      Dies gilt sowohl für den Benutzernamen und das Kennwort, als auch für Sicherungen der individuellen Schlüsseldatei zur
	      Ver- und Entschlüsselung der Kommunikation.
	\item Der Nutzer verpflichtet sich, über die Anwendung keine Inhalte weiterzuleiten, die rechtswidrig, insbesondere rassistisch,
	      pornografisch, beleidigend oder verleumderisch sind oder die Rechte Dritter, insbesondere Urheberrechte bzw. urheber-
	      rechtliche Nutzungsrechte verletzen.
	\item Der Nutzer ist verpflichtet, niemanden mit Kommunikationsversuchen zu belästigen.
	\item Der Nutzer hat alle Handlungen zu unterlassen, die geeignet sind, die Funktionalität der Anwendung zu beeinträchtigen,
	      insbesondere übermäßig zu belasten.
	\item Der Nutzer verpflichtet sich, sämtliche Systeme, mit denen auf Inhalte und Kommunikationsfunktionen der Webseite /
	      Anwendung zugegriffen wird, durch Sicherheitsupdates aktuell zu halten.
	\item Der Nutzer trägt dafür Sorge, sämtliche Systeme und Endgeräte, mit denen auf Inhalte und Kommunikationsfunktionen der
	      Anwendung zugegriffen wird, durch geeignete Schutzmaßnahmen vor unberechtigten Zugriffen zu schützen.
\end{itemize}

\section{Änderung des Leistungsumfangs}
helpwave behält sich vor, die angebotene Anwendung sowie zugehörige Dienste jederzeit inhaltlich, grafisch und/oder funktionell durch Updates zu verändern oder zu erweitern.
\section{Zeitliche Verfügbarkeit}
helpwave wird sich bemühen, die angebotene Anwendung und zugehörige Dienste möglichst unterbrechungsfrei zum Abruf
anzubieten. Der Nutzer erkennt an, dass helpwave auch bei aller Sorgfalt eine zeitlich ununterbrochene Verfügbarkeit seiner
Dienste technisch nicht garantieren kann. Insbesondere behält sich helpwave vor, seine Dienste aus Wartungs-, Sicherheits-
oder Kapazitätsgründen nach dem Ermessen von helpwave vorübergehend einzuschränken oder auszusetzen. helpwave wird
den Kunden über größere geplante Wartungsarbeiten (länger als 30 Minuten) eine Woche im Voraus informieren.
\section{Haftung und Gewährleistungsausschluss von helpwave}
helpwave haftet nur bei Vorsatz und grober Fahrlässigkeit. Bei leichter Fahrlässigkeit haftet helpwave nur für Verletzungen
einer wesentlichen vertraglichen Verpflichtung, die erforderlich ist, damit der Vertrag ordnungsgemäß umgesetzt werden
kann und auf die sich der Nutzer normalerweise verlassen kann. Die Haftung ist auf die vorhersehbaren und vertragstypischen Schäden beschränkt. Insbesondere wird die Haftung für Datenverlust auf den typischen Wiederherstellungsaufwand
beschränkt, der bei regelmäßiger und gefahrentsprechender Anfertigung von Sicherungskopien eingetreten wäre.
Die unter vorstehendem Absatz enthaltenen Haftungsbeschränkungen umfassen auch etwaige Ansprüche des Nutzers auf
den Ersatz solcher Aufwendungen, die er im Vertrauen auf den Erhalt einer vertragsgemäßen Leistung gemacht hat und
billigerweise machen durfte.
Hingegen haftet helpwave gegenüber Nutzern uneingeschränkt für Schäden aus der Verletzung des Lebens, des Körpers
oder der Gesundheit. Darüber hinaus haftet helpwave uneingeschränkt für Schäden, die von der Haftung nach zwingend
geltenden gesetzlichen Vorschriften umfasst werden sowie im Fall der etwaigen Übernahme einer Garantie durch helpwave.

\section{Geistiges Eigentum}
Alle Urheber-, insbesondere Datenbank-, Marken-, Geschmacksmuster- und andere Rechte zum Schutze geistigen Eigentums
(ebenso die Anordnung und die Darstellung der Webseite) liegen bei helpwave oder ihren Lizenzgebern. Ohne die vorherige
schriftliche Zustimmung dürfen die Inhalte der Webseite oder der zugrunde liegende Software-Code weder ganz oder teilweise vervielfältigt, bearbeitet, verbreitet, öffentlich wiedergegeben noch in sonstiger Form genutzt werden.

\section{Sonstiges}
helpwave behält sich vor, diese Nutzungsbedingungen jederzeit ohne Nennung von Gründen zu verändern oder zu erweitern,
es sei denn, das ist für den Nutzer nicht zumutbar. helpwave wird den Nutzer über jede Änderung der Nutzungsbedingungen rechtzeitig benachrichtigen. Widerspricht der Nutzer der Geltung der neuen Nutzungsbedingungen nicht innerhalb von
sechs (6) Wochen nach der Benachrichtigung, gelten die geänderten Nutzungsbedingungen als vom Nutzer angenommen.
helpwave wird den Nutzer in der Benachrichtigung auf sein Widerspruchsrecht und die Bedeutung der Widerspruchsfrist
hinweisen.
Soweit nichts anderes vereinbart ist, kann der Nutzer alle Erklärungen an helpwave per E-Mail oder per Brief an helpwave
übermitteln. helpwave kann Erklärungen gegenüber dem Nutzer per E-Mail oder Brief an die Adressen übermitteln, die der
Nutzer als aktuelle Kontaktdaten in seinem Konto angegeben hat.
Sollten einzelne Regelungen dieser Nutzungsbedingungen unwirksam sein oder werden, wird dadurch die Wirksamkeit der
übrigen Regelungen nicht berührt. \\
Erfüllungsort und Gerichtsstand ist der Sitz von helpwave.
Es gilt deutsches Recht.

\end{document}
