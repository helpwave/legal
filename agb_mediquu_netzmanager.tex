\documentclass[10pt]{article}
\usepackage[margin=2cm]{geometry}
\usepackage[ngerman]{babel}

\usepackage[inkscapeformat=png]{svg}
\usepackage{fancyhdr}

\usepackage{lastpage}

\usepackage{inter}
\renewcommand\familydefault{\sfdefault}

\usepackage[onehalfspacing]{setspace}
\usepackage[iso,german]{isodate}

\usepackage{xcolor}
\usepackage{hyperref}

\setcounter{tocdepth}{1}
\setlength\parindent{0pt}

\usepackage{titlesec}
\titleformat{\subsection}[runin]{\normalfont\bfseries}{\thesubsection}{0.5em}{}[]
\titleformat{\subsubsection}[runin]{\normalfont\bfseries}{\thesubsubsection}{0.5em}{}[]

\def\Year{\expandafter\YEAR\the\year}
\def\YEAR#1#2#3#4{#1#2#3#4}

\fancyhead{}
\pagestyle{fancy}
\renewcommand{\headrulewidth}{0pt}
\fancyfoot{}
\fancyfoot[L]{\textcolor{gray}{\thepage\space/\space\pageref{LastPage}}}
\fancyfoot[R]{\textcolor{gray}{\href{https://helpwave.de}{helpwave.de}}}

\begin{document}

\begin{center}
	\includesvg[width=0.2\linewidth]{logo.svg}\\[4ex]
	{\Huge \sffamily
	Nutzungsbedingungen \\
	\huge mediQuu Netzmanager}\\[2ex]

	\vspace{1em}
	{\large
		helpwave GmbH \\
		Jülicher Straße 209d \\
		52070 Aachen \\
		HRB 27480}\\
	\vspace{1em}
	Stand: \today
\end{center}

\vspace{2ex}

\tableofcontents

\newpage

\section{Verwendete Begrifflichkeiten}
Im Rahmen dieser Nutzungsbedingungen wird zwischen „Kunde“ und „Nutzern“ unterschieden. „Kunde“ ist der Vertragspartner von helpwave, der die Anwendung „mediQuu Netzmanager“ (im Folgenden „Anwendung“ oder „Netzmanager“ genannt) lizenziert. Als „Nutzer“ werden Mitarbeiter des Kunden bezeichnet, die von diesem autorisiert sind, die Anwendung operativ zu verwenden.
\section{Geltung der Nutzungsbedingungen}
Diese Nutzungsbedingungen, die gelegentlichen Veränderungen unterliegen, gelten für alle helpwave-Dienstleistungen rund
um die Anwendung, die unmittelbar oder mittelbar (d.h. über Dritte) über das Internet, jegliche Art von mobilen Endgeräten
oder per E-Mail zur Verfügung gestellt werden. Mit dem Zugriff auf die (mobile) Webseite von mediQuu und die Nutzung der
Anwendung, ganz gleich durch welche Plattform, bestätigt der Kunde, dass er die unten aufgeführten Nutzungsbedingungen gelesen und verstanden hat und diesen zustimmt.
\section{Zustandekommen des Nutzungsvertrags}
Der Kunde schließt einen Vertrag über die Nutzung der Anwendung mit der helpwave GmbH. Über eine passwortgesicherte Webanwendung erhält der Kunde Zugriff auf die Anwendung. Durch das Herunterladen und / oder das Installieren und / oder den webbasierten Gebrauch der Software erklärt sich der Kunde mit diesen Nutzungsbedingungen einverstanden und erkennt diese ohne Einschränkung verbindlich an.
\section{Art und Umfang der Anwendung}
helpwave stellt dem Kunden mit dem Netzmanager eine Plattform zur Verwaltung von und zur Kommunikation in ärztlichen
Praxisnetzwerken zur Verfügung.
Die Anwendung bietet dem Kunden und den Nutzern über das Modul „Mitgliederverwaltung“ die Möglichkeit, Mitglieder
und Kontakte des Praxisnetzes zu speichern und zu verschlagworten. Die zu jedem Kontakt gespeicherten Grunddaten umfassen den Namen, Organisationsnamen, Anschrift sowie Kontaktdaten (Telefon, Fax, E-Mail, Homepage, Kontaktpräferenz).
Die Grunddaten können durch eigene, selbst definierbare Datenfelder vom Kunden ergänzt werden. Verpflichtend sind aus
technischen Gründen nur der Name und Ort anzugeben. Alle weiteren Daten sind optional.
Die hinterlegten Kontaktdaten können in dem Modul „Messenger“ genutzt werden, um die Mitglieder auf einem oder mehreren Kommunikationskanälen zu kontaktieren. Die unterstützten Wege sind E-Mail, Fax, SMS und Push-Benachrichtigung.
Die versendeten Nachrichten werden – je nach gewähltem Kanal – per E-Mail über Server Hetzner, per Fax über
Simple-Fax, per Push-Nachricht über die Plattform-Anbieter Apple und Google und per SMS über LOX24 an die Mitglieder
übermittelt.
Versendete Nachrichten werden im Netzmanager-Archiv gespeichert und sind dort für die Nutzer einsehbar. Rückläufer von
nicht-zustellbaren E-Mails werden ebenfalls im System gespeichert und sind für die Nutzer abrufbar. Weiterhin können die
Nutzer über den Netzmanager über eine eigene, integrierte Cloud-Lösung Dokumente für ihre Praxisnetz-Mitglieder zur
Verfügung stellen, welche diese über einen geschützten Bereich in der mediQuu-App (verfügbar im App Store von Apple
oder dem Play Store von Google) oder über die Webadresse intranet.mediquu.de abrufen können. Nutzer können weiterhin
im Bereich „Umfragen“ digitale Fragebögen erstellen. Eine Umfrage umfasst dabei einen Titel und eine Kurzbeschreibung,
sowie definierbare Fragen mit standardisierten Antwortmöglichkeiten. Die Umfrage kann über das „Messenger-Modul“
direkt an die jeweiligen Empfänger über einen der zuvor genannten Kommunikationskanäle versandt werden. In Analogie
zur Umfrage können auch „Veranstaltungsanmeldungen“ konzipiert werden. Eine solche Anmeldung besteht ebenfalls aus einem Titel, einer Beschreibung, einem Ort und Datum sowie selbst definierbaren Eingabefeldern. Auch die Veranstaltungsanmeldung kann nach ihrer Erstellung über den Messenger kommuniziert werden. Weiterhin haben Nutzer die Möglichkeit,
einen zentralen „Terminkalender“ zu pflegen, den Praxisnetz-Mitglieder über die mediQuu-App abrufen können, ebenso wie
eine zentrale „Neuigkeiten-Liste“, über die Neuigkeiten des Praxisnetzes für die Mitglieder bereitgestellt werden können.
Über das Netzmanager-Modul „Intranet“ können die Nutzer dabei definieren, welche Daten ihre Praxisnetz-Mitglieder über
die mediQuu-App oder die Webadresse intranet.mediquu.de abrufen können.

\section{Entgelt und Zahlungsbedingungen}
helpwave stellt dem Kunden die Nutzung des Netzmanagers gegen eine monatliche Lizenzgebühr zur Verfügung. helpwave
behält sich Gebührenänderungen vor. Geänderte Gebühren, die für das jeweilige Vertragsverhältnis entscheidend sind, werden dem Nutzer mindestens sechs Wochen vor Inkrafttreten per E-Mail oder postalisch mitgeteilt. Preiserhöhungen bewirken ein außerordentliches Kündigungsrecht mit einer Frist von vier Wochen zum Zeitpunkt des Inkrafttretens der Erhöhung.
Die Rechnungen von helpwave sind innerhalb von 14 Tagen und ohne Abzug nach Rechnungseingang zu überweisen. Eine
Aufrechnung des Kunden gegenüber Forderungen von helpwave mit anderen als unbestritten oder rechtskräftig festgestell-
ten Forderungen ist ausgeschlossen, es sei denn, die Gegenforderung und die aufgerechnete Hauptforderung sind synallagmatisch miteinander verknüpft.
Befindet sich der Kunde mit dem Ausgleich einer Forderung im Verzug, ist helpwave berechtigt, Verzugszinsen in Höhe von
drei Prozentpunkten über dem Basiszinssatz gemäß § 288 BGB zu verlangen. Die Geltendmachung eines weiteren Verzugs-
schadens bleibt davon unberührt.

\section{Abfrage und Änderung von personenbezogenen Daten durch Mitglieder des Kunden}
Die in dem Modul „Mitgliederverwaltung“ gespeicherten Daten werden durch den Kunden bzw. den Nutzern verarbeitet.
Mitglieder eines Praxisnetzes, deren Daten im Netzmanager gespeichert sind, können deshalb ihre persönlichen Daten beim
Kunden direkt abfragen, ändern oder löschen lassen. Jeder Betroffene kann dies über eine Mitteilung an den Kunden beantragen. Der Kunde kann helpwave mit der Änderung oder Löschung aller sowie von Teilen der im Netzmanager gespeicherten Daten des Praxisnetzwerkes beauftragen. Die unter nachstehender Ziffer getroffenen Bestimmungen zur Speicherung von Daten über den Zeitpunkt der Beendigung des mit dem Nutzer abgeschlossenen Nutzungsvertrages hinaus bleiben unberührt.

\section{Datenschutz}
Zur Nutzung des Netzmanagers benötigt der Nutzer Zugangsdaten. Diese werden nach Weisung des Kunden von helpwave
angelegt und an den Kunden übermittelt. Der Kunde stellt sicher, dass nur für autorisierte Nutzer Zugangsdaten angefordert werden. Die Identifikation des Nutzers erfolgt durch Benutzername und Passwort. Zusätzlich wird durch helpwave eine
Absender-E-Mail-Adresse des Kunden für den Versand von Nachrichten hinterlegt. Die im Netzmanager gespeicherten Daten des Kunden werden auf gesicherten und zertifizierten Serversystemen der Deutschen Telekom an einem deutschen
Standort gespeichert. Der Transport der Daten vom Endgerät des autorisierten Nutzers zum Server erfolgt SSL-verschlüsselt.
Innerhalb des Netzmanagers können die autorisierten Nutzer Daten zu Praxisnetzmitgliedern hinterlegen. Diese umfassen
standardmäßig den Titel und Namen, die Anschrift sowie Kontaktdaten wie Telefon, Fax, E-Mail, Internetseite und Kontakt-
präferenz. Mit Ausnahme von Name und Ort sind alle Angaben optional. Auf Wunsch kann der Kunde weitere Felder für das
Netz anlegen, in denen entsprechende Daten eingegeben werden können. helpwave ist für die im Netzmanager enthaltenen Daten Auftragsverarbeiter des Kunden. Eine entsprechende Auftragsverarbeitungsvereinbarung (AVV) kann ein vertretungsberechtigter Mitarbeiter des Kunden bei der ersten Anmeldung in der Anwendung elektronisch abschließen. Eine Nutzung des Netzmanagers ohne den Abschluss einer AVV ist nicht möglich. \\

Weitere Hinweise zum Datenschutz finden Sie hier: \href{https://cdn.helpwave.de/privacy.html}{cdn.helpwave.de/privacy.html}

\section{Zulässige Nutzung}
Der Netzmanager darf ausschließlich für die folgenden Zwecke genutzt werden: (1.) Anzeigen der Anwendung, (2.) Prüfen
von Informationen zu dem Leistungsangebot, (3.) Speicherung und Verarbeitung von Mitgliedsdaten sowie Kommunikation
mit Mitgliedern und (4.) Nutzung weiterer Funktionen, die in der Anwendung verfügbar gemacht werden.
Jede andere Nutzung des Netzmanager ist ohne vorherige schriftliche Zustimmung von helpwave untersagt. Die Vervielfältigung, Bearbeitung, Verbreitung, öffentliche Wiedergabe oder jede sonstige Form der Nutzung von Inhalten dieser Webseite
zu gewerblichen Zwecken ist unzulässig. Ebenso untersagt ist die Verwendung automatisierter Systeme oder automatisierter Software zur Extraktion von Inhalten der Anwendung. Auch ist jeder Zugriff auf Inhalte der Anwendung unzulässig, der
nicht über die Benutzeroberfläche dieser erfolgt. Unbeschadet der Geltendmachung sonstiger Rechte behält sich helpwave
vor, den Zugang zum Netzmanager jederzeit zu sperren, wenn gegen diese Nutzungsbedingungen verstoßen wird. Unzulässig ist die Verarbeitung von sensiblen personenbezogenen Daten, wie bspw. Gesundheitsdaten.

\section{Pflichten des Kunden und Nutzers}
Der Kunde verpflichtet sich, nur autorisierten Nutzern Zugang zum Netzmanager zu ermöglichen. Weiterhin ist der Kunde verpflichtet, die Nutzer über die zulässige Nutzung des Netzmanagers zu informieren und zu verpflichten. Der Kunde
verpflichtet sich und seine Nutzer, über den Netzmanager keine Inhalte weiterzuleiten, die rechtswidrig, insbesondere rassistisch, pornografisch, beleidigend oder verleumderisch sind oder die Rechte Dritter, insbesondere Urheberrechte bzw. urheberrechtliche Nutzungsrechte verletzen. Der Kunde verpflichtet sich und seine Nutzer darüber hinaus, niemanden mit
Kommunikationsversuchen zu belästigen. Der Kunde und seine Nutzer haben alle Handlungen zu unterlassen, die geeignet
sind, die Funktionalität des Netzmanagers zu beeinträchtigen, insbesondere übermäßig zu belasten. Der Kunde verpflichtet
sich und seine Nutzer, digital erfasste Informationen stets auf ihre Richtigkeit hin zu überprüfen und ggf. zu korrigieren und
Zugangsdaten zum Netzmanager streng vertraulich zu behandeln und nicht weiterzugeben. Dies gilt insbesondere für den
Benutzernamen und das Kennwort.

\section{Änderung des Leistungsumfangs}
helpwave behält sich vor, die von helpwave angebotenen Dienste und Anwendungen jederzeit inhaltlich, grafisch und/oder
funktionell durch Updates zu verändern oder zu erweitern.

\section{Zeitliche Verfügbarkeit}
helpwave wird sich bemühen, die angebotenen Dienste und Lösungen möglichst unterbrechungsfrei zum Abruf anzubieten.
Der Nutzer erkennt an, dass helpwave auch bei aller Sorgfalt eine zeitlich ununterbrochene Verfügbarkeit seiner Dienste
technisch nicht garantieren kann. Insbesondere behält sich helpwave vor, seine Dienste aus Wartungs-, Sicherheits- oder
Kapazitätsgründen nach dem Ermessen von helpwave vorübergehend einzuschränken oder auszusetzen. helpwave wird den
Kunden über größere geplante Wartungsarbeiten (länger als 30 Minuten) eine Woche im Voraus informieren.

\section{Gewährleistung}
helpwave prüft die Anwendung Netzmanager auf Funktionalität und Lauffähigkeit. Fehler in Dateien, Programmen oder Programmteilen werden von helpwave behoben. Dies gilt auch, wenn nur Teile fehlerhaft programmiert sind und es dadurch zu
Problemen kommt. helpwave erhält in einem solchen Fall die Möglichkeit, nach eigenem Ermessen nachzubessern oder kostenlosen Ersatz zu liefern. Sollten drei Nachbesserungsversuche fehlschlagen, so ist der Nutzer nach seiner Wahl berechtigt,
Minderung zu verlangen. Für Ausfälle im Internet, die dazu führen, dass die Webseiten nicht abrufbar sind, kann helpwave
nicht haftbar gemacht werden. Nimmt der Nutzer selbst Eingriffe am Quelltext vor, erlischt jeglicher Gewährleistungs- und
Haftungsanspruch.
\section{Haftung}
helpwave haftet nur bei Vorsatz und grober Fahrlässigkeit. Bei leichter Fahrlässigkeit haftet helpwave nur für Verletzungen
einer wesentlichen vertraglichen Verpflichtung, die erforderlich ist, damit der Vertrag ordnungsgemäß umgesetzt werden
kann und auf die sich der Kunde normalerweise verlassen kann. Die Haftung ist auf die vorhersehbaren und vertragstypischen Schäden beschränkt. Insbesondere wird die Haftung für Datenverlust auf den typischen Wiederherstellungsaufwand
beschränkt, der bei regelmäßiger und gefahrentsprechender Anfertigung von Sicherungskopien eingetreten wäre.
Die unter vorstehendem Absatz enthaltenen Haftungsbeschränkungen umfassen auch etwaige Ansprüche des Kunden auf
den Ersatz solcher Aufwendungen, die er im Vertrauen auf den Erhalt einer vertragsgemäßen Leistung gemacht hat und
billigerweise machen durfte.
Hingegen haftet helpwave uneingeschränkt für Schäden aus der Verletzung des Lebens, des Körpers oder der Gesundheit.
Darüber hinaus haftet helpwave uneingeschränkt für Schäden, die von der Haftung nach zwingend geltenden gesetzlichen
Vorschriften umfasst werden sowie im Fall der etwaigen Übernahme einer Garantie durch helpwave.
\section{Geistiges Eigentum}
Alle Urheber-, insbesondere Datenbank-, Marken-, Geschmacksmuster- und andere Rechte zum Schutze geistigen Eigentums
(ebenso die Anordnung und die Darstellung der Webseite) liegen bei helpwave oder ihren Lizenzgebern. Ohne die vorherige
schriftliche Zustimmung dürfen Sie die Inhalte der Webseite oder der zugrunde liegende Software-Code weder ganz oder
teilweise vervielfältigen, bearbeiten, verbreiten, öffentlich wiedergegeben noch in sonstiger Form nutzen.
\section{Sonstiges}
helpwave behält sich vor, diese Nutzungsbedingungen jederzeit ohne Nennung von Gründen zu ändern, es sei denn, das ist
für den Nutzer nicht zumutbar. helpwave wird den Kunden über Änderungen der Nutzungsbedingungen rechtzeitig benachrichtigen. Widerspricht der Nutzer der Geltung der neuen Nutzungsbedingungen nicht innerhalb von sechs (6) Wochen nach
der Benachrichtigung, gelten die geänderten Nutzungsbedingungen als vom Kunden angenommen. helpwave wird den Kunden in der Benachrichtigung auf sein Widerspruchsrecht und die Bedeutung der Widerspruchsfrist hinweisen.
Soweit nichts anderes vereinbart ist, kann der Kunde alle Erklärungen an helpwave per E-Mail oder per Brief an helpwave
übermitteln. helpwave kann Erklärungen gegenüber dem Kunden per E-Mail oder Brief an die Adressen übermitteln, die der
Kunde als aktuelle Kontaktdaten in seinem Konto angegeben hat.
Sollten einzelne Regelungen dieser Nutzungsbedingungen unwirksam sein oder werden, wird dadurch die Wirksamkeit der
übrigen Regelungen nicht berührt. \\
Erfüllungsort und Gerichtsstand ist der Sitz von helpwave.
Es gilt deutsches Recht.

\end{document}
